\lstset{escapeinside={!}{!}}
\begin{tikzpicture}[remember picture]
\node(code) [anatomy] at (0,0){%
\begin{lstlisting}
public !\iPart{class}{class}! !\cPart{className}{HelloWorld}!
{
    !\extremPoint{mainLeft}[most top]!public static void main(String[] argv)
    {
        !\extremPoint{left}[most top]\iPart{assign}{
          \bgcode{// Prints "Hello World" in the terminal window}}
          \extremPoint{fnR} \extremPoint{mR}!
        !\iPart{fnCall}{System.out.print( "Hello World");}\extremPoint{mostBottom}[most bottom]!
    }!\extremPoint{mainBottom}[most bottom]!
}
\end{lstlisting}
};

\fitExtrem{classBody}{(mainLeft) (mR) (mainBottom)}
\fitExtrem{functionBody}{(left) (fnR) (mostBottom)}


\codeAnnotation{fileNameText} (1.5,5) {text file named \texttt{HelloWorld.java}}
\codeAnnotation{classNameText} (3.5,4.25) {name}
\codeAnnotation{classBodyText} (6.5,3.6) {\texttt{main()} method}
\codeAnnotation{functionBodyText} (2.5,-0.5) {body}
\codeAnnotation{statement} (8,0) {statements}

\draw[->,annotation] (fileNameText) -- (class);
\draw[->,annotation] (classNameText) -- (className);
\draw[->,annotation] (classBodyText.south west) -- (classBody);
\draw[->,annotation] (functionBodyText) -- (functionBody);
\draw[->,annotation] (statement) -- (assign.353);
\draw[->,annotation] (statement) -- (fnCall.350);
\end{tikzpicture}
